%%%%%%%%%%%%%%%%%%%%%%%%%%%%%%%%%%%%%%%%%%%%%%%%%%%%%%%%%%%%%%%%%%%%%%%%%%%
%
% Generic template for TFC/TFM/TFG/Tesis
%
% $Id: introduccion.tex,v 1.6 2014/02/11 11:00:06 macias Exp $
%
% By:
%  + Javier Mac�as-Guarasa.
%    Departamento de Electr�nica
%    Universidad de Alcal�
%  + Roberto Barra-Chicote.
%    Departamento de Ingenier�a Electr�nica
%    Universidad Polit�cnica de Madrid
%
% Based on original sources by Roberto Barra, Manuel Oca�a, Jes�s Nuevo,
% Pedro Revenga, Fernando Herr�nz and Noelia Hern�ndez. Thanks a lot to
% all of them, and to the many anonymous contributors found (thanks to
% google) that provided help in setting all this up.
%
% See also the additionalContributors.txt file to check the name of
% additional contributors to this work.
%
% If you think you can add pieces of relevant/useful examples,
% improvements, please contact us at (macias@depeca.uah.es)
%
% Copyleft 2013
%
%%%%%%%%%%%%%%%%%%%%%%%%%%%%%%%%%%%%%%%%%%%%%%%%%%%%%%%%%%%%%%%%%%%%%%%%%%%

\chapter{Plataforma rob�tica base}
\label{cha:plataforma_robotica_base}

\begin{FraseCelebre}
  \begin{Frase}
    Desocupado lector, sin juramento me podr�s creer que quisiera que este
    libro [...] fuera el m�s hermoso, el m�s gallardo y m�s discreto que
    pudiera imaginarse\footnote{Tomado de ejemplos del proyecto \texis{}.}.
  \end{Frase}
  \begin{Fuente}
    Miguel de Cervantes, Don Quijote de la Mancha
  \end{Fuente}
\end{FraseCelebre}


\section{Introducci�n}

\section{Arquitectura mec�nica}
\label{sec:arquitectura_mecanica}

\section{Arquitectura hardware}
\label{sec:arquitectura_hardware}

\section{Arquitectura software}
\label{sec:arquitectura_software}


\section{Din�mica de un sistema de tracci�n diferencial}
\label{sec:dinamica_sma_traccion_diferencial}

	\subsection{Coeficiente de adherencia de las ruedas, c�lculo experimental}
	\subsection{Determinaci�n de la aceleraci�n m�xima}
	\subsection{Perfil de velocidad trapezoidal}
	\subsection{Leyes mec�nicas y t�rmicas de un motor DC}
	\subsection{Condici�n de deslizamiento y dimensionamiento de la reductora}
	\subsection{Elecci�n de la velocidad y la aceleraci�n}

\section{Control en posici�n de tipo polar}
\label{sec:control_posicion_tipo_polar}

	\subsection{Tipos de controladores}
	\subsection{Ajuste del controlador PID}

\section{Posicionamiento relativo mediante odometr�a}
\label{sec:posicionamiento_relativo_mediante_odometria}

	\subsection{Medida y correcci�n de errores sistem�ticos}
	\subsection{Compensaci�n de la fuerza centr�peta}

\section{Posicionamiento absoluto mediante balizas}
\label{sec:posicionamiento_absoluto_mediante_balizas}

\section{Detecci�n de bloqueos mec�nicos}
\label{sec:deteccion_bloqueos_mecanicos}

\section{Generaci�n de trayectorias}
\label{sec:generacion_de_trayectorias}

\section{Evitaci�n de obst�culos}
\label{sec:evitacion_de_obstaculos}

	\subsection{Sistemas detecci�n de obst�culos}
	\subsection{Detecci�n de oponentes mediante baliza tipo faro}





%%% Local Variables:
%%% TeX-master: "../book"
%%% End:
