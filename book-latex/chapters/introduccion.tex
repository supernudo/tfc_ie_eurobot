%%%%%%%%%%%%%%%%%%%%%%%%%%%%%%%%%%%%%%%%%%%%%%%%%%%%%%%%%%%%%%%%%%%%%%%%%%%
%
% Generic template for TFC/TFM/TFG/Tesis
%
% $Id: introduccion.tex,v 1.6 2014/02/11 11:00:06 macias Exp $
%
% By:
%  + Javier Mac�as-Guarasa. 
%    Departamento de Electr�nica
%    Universidad de Alcal�
%  + Roberto Barra-Chicote. 
%    Departamento de Ingenier�a Electr�nica
%    Universidad Polit�cnica de Madrid   
% 
% Based on original sources by Roberto Barra, Manuel Oca�a, Jes�s Nuevo,
% Pedro Revenga, Fernando Herr�nz and Noelia Hern�ndez. Thanks a lot to
% all of them, and to the many anonymous contributors found (thanks to
% google) that provided help in setting all this up.
%
% See also the additionalContributors.txt file to check the name of
% additional contributors to this work.
%
% If you think you can add pieces of relevant/useful examples,
% improvements, please contact us at (macias@depeca.uah.es)
%
% Copyleft 2013
%
%%%%%%%%%%%%%%%%%%%%%%%%%%%%%%%%%%%%%%%%%%%%%%%%%%%%%%%%%%%%%%%%%%%%%%%%%%%

\chapter{Introducci�n al proyecto}
\label{cha:introduccion}

\begin{FraseCelebre}
  \begin{Frase}
    Desocupado lector, sin juramento me podr�s creer que quisiera que este
    libro [...] fuera el m�s hermoso, el m�s gallardo y m�s discreto que
    pudiera imaginarse\footnote{Tomado de ejemplos del proyecto \texis{}.}.
  \end{Frase}
  \begin{Fuente}
    Miguel de Cervantes, Don Quijote de la Mancha
  \end{Fuente}
\end{FraseCelebre}


\section{Presentaci�n}
\label{sec:presentacion}

Este proyecto trata sobre el desarrollo de robots para la competici�n de robots Eurobot \cite{eurobot}. Pretende ser un manual de referencia para todo aquel que quiera constuir un robot para participar en Eurobot. Todo lo que se expone en este libro esta se sustenta en la experiencia adquirida en el desarrollo de robots participantes en Eurobot entre los a�os 2003 y 2015. Especialmente en el periodo entre 2010 y 2015.

\section{La competici�n Eurobot}
\label{sec:la_competicion_eurobot}


Vision desde dentro, reto, aplicacion conocimientos, trabajo en equipo, gestion de recursos, organizaci�n...

\section{Rob�tica educativa}
\label{sec:robotica_educativa}

\section{El equipo Eurobotics Engineering}
\label{sec:equipo_eurobotics}

\section{A hombros de gigantes}
\label{sec:a_hombros_de_gigantes

\section{Motivaci�n y objetivos}
\label{sec:motivacion_y_objetivos}

El objetivo fundamental de este proyecto es la divulgaci�n del conocimiento adquirido en la realizaci�n de robots que compiten en Eurobot.

Los objetivos espec�ficos de este proyecto son los siguientes:

\begin{itemize}
\item Reflexionar y analizar diferentes aspectos indirectamente relacionados con el desarrollo de ingenier�a (el equipo de trabajo, organizaci�n, ...)
\item Analizar y describir los fundamentos de ingenier�a impl�citos (din�mica, cinem�tica, sistemas de control, ...).
\item Describir las partes fundamentales del desarrollo (movimiento, navegaci�n, posicionamiento, estrategia ...) y dise�o un robot de Eurobot (mec�nica, electr�nica, programaci�n, ...).
\item Presentar implementaciones y dise�os reales de cada una de las partes fundamentales de un robot de Eurobot. 
\item Documentar todas aquellos dise�os, desarrollos o algoritmos realizados que puedan servir de punto de partida o ser reutilizados en futuros desarrollos (dise�os electr�nicos, librer�as software, herramientas de simulaci�n, ...). 
\end{itemize}

\section{Organizaci�n del libro}
\label{sec:organizacion_del_libro}




%%% Local Variables:
%%% TeX-master: "../book"
%%% End:
